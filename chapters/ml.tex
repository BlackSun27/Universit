\chapter{Esercizi ML}

\section{Manipolazione di interi}

\subsection{Fattoriale}

Si scriva il fattoriale di un numero (definito come n!=n*(n-1)! e 0! = 1! = 1).

\begin{lstlisting}
    fun fatt(n) = if n=0 then 1 else n*fatt(n-1);
\end{lstlisting}

\subsection{Sequenza di Fibonacci}

La sequenza di Fibonacci è la somma di numeri definita come: 
\newline
Fib(n) = Fib(n-1) + Fib(n-2)
\newline
Dove Fib(0) = Fib(1) = 1.

\begin{lstlisting}
    fun fib_sum(n) = if n=0 then 1 
    else if n=1 then 1 
    else fib_sum(n-1)+fib_sum(n-2);
\end{lstlisting}

\section{Manipolazione liste}
Le liste sono state definite nel modo seguente:

\begin{lstlisting}
    datatype List = Empty | nodo of int * List;
\end{lstlisting}

\subsection{Conta}

Si conti la lunghezza di una lista.

\begin{lstlisting}
    fun conta (Empty) = 0
= | conta (nodo(x,l)) = conta(l) + 1;
\end{lstlisting}

\subsection{insert}

Funzione per inserire elementi in una lista.

\begin{lstlisting}
    fun insert(x,Empty) = nodo(x,Empty)
    | insert(x: int, l: List) = nodo(x,l);
\end{lstlisting}

\subsection{insertAt}

Funzione per inserire un elemento nella lista in un punto specifico della lista (la testa della lista parte sempre da 0).

\begin{lstlisting}
    fun insertAt(x: int, 0, l: List) = nodo(x, l)
  | insertAt(x: int, n, Empty) = Empty
  | insertAt(x: int, n, nodo(h, t)) = nodo(h, insertAt(x, n-1, t));

\end{lstlisting}

\subsection{max_value}

Funzione che calcoli l'elemento più grande presente in una lista.

\begin{lstlisting}
    fun max_value(Empty) = 0
=   | max_value(nodo(x, Empty)) = x
=   | max_value(nodo(x, xs)) =
=       let
=         val max_tail = max_value(xs)
=       in
=         if x > max_tail then x else max_tail
=       end;
\end{lstlisting}

\subsection{min-value}

Funzioni che calcoli l'elemento più piccola in una lista.

\begin{lstlisting}
    fun min_value(Empty) = 0
    | min_value(nodo(x, Empty)) = x
    | min_value(nodo(x,xs)) = 
	let
	   val min_tail = min_value(xs)
	in
	   if x < min_tail then x else min_tail
	end;
\end{lstlisting}

\subsection{sum-list}

Somma tutti gli elementi di una lista.

\begin{lstlisting}
    fun sum_list(Empty) = 0
    = | sum_list(nodo(x,l)) = x + sum_list(l);
\end{lstlisting}

\subsection{count-occurrencies}

Conta quante volte si ripete un elemento in una lista.

\begin{lstlisting}
    fun count_occurrences(_, Empty) = 0
    | count_occurrences(x, nodo(h, t)) =
      if x = h then 1 + count_occurrences(x, t)
      else count_occurrences(x, t);
\end{lstlisting}

\subsection{contains}

Verifica se un elemento è contenuto nella lista.

\begin{lstlisting}
    fun contains(_,Empty) = false
    = | contains(x, nodo(h,t)) =
    = if x=h then true else contains(x,t);
\end{lstlisting}

\subsection{remove-element}

Elimina un elemento da una lista.

\begin{lstlisting}
    fun remove_element(_, Empty) = Empty
    = | remove_element(x, nodo(h,t)) =
    = if x=h then remove_element(x,t) 
        else nodo(h, remove_element(x,t));
\end{lstlisting}

\subsection{reverse}

Invertire gli elementi di una lista.

\begin{lstlisting}
    fun reverse l =
    = let
    = fun rev(Empty,acc) = acc
    = |rev(nodo(x,xs),acc) = rev(xs, nodo(x,acc))
    = in
    = rev(l,Empty)
    = end;
\end{lstlisting}

\subsection{concat}

Concatenare due liste in una.

\begin{lstlisting}
    fun concat(Empty, l2) = l2 
    = |concat(nodo(h,t), l2) = nodo(h, concat(t,l2));
\end{lstlisting}

\subsection{nth-element}

Trova l'ennesimo elemento di una lista partendo da 0.

\begin{lstlisting}
    fun nth_element(_,Empty) = 0
    = |nth_element(0, nodo(x,_)) = x
    = |nth_element(n, nodo(h,t)) = nth_element(n-1,t);
\end{lstlisting}

\subsection{split-list}

Divide una lista in 2 (di lunghezza l), data una lunghezza (n) di partenza in modo che la lista risultante 1 abbia gli elementi che vanno da 0 a n-1 e la seconda lista abbia gli elementi che vanno da n a l.

\begin{lstlisting}
    fun split_list(0,l) = (Empty,l) //1
    = |split_list(_,Empty) = (Empty,Empty) //2
    = |split_list(n, nodo(x,xs))= 
    = let
    = val (l1,l2) = split_list(n-1,xs) //3
    = in
    = (nodo(x,l1),l2) //4
    = end;
\end{lstlisting}

\subsubsection{Spiegazione}

Come funziona split\_list?

//1 -> La posizione per dividere in 2 la lista di partenza è nulla, quindi il risultato darà una lista vuota (Empty) e un'altra lista, che è quella di partenza.


//2 -> Posizione qualunque, ma lista di partenza vuota, il risultato della funzione è 2 liste vuote.


//3 -> dichiaro 2 liste e man mano che n-1 diminuisce (arrivando a 0), viene riempita l1 con gli elementi di nodo(x,xs) che trova, appena n=0 si torna al caso //1.


//4 -> per far partire il tutto (e popolare l1, che diventerà la nuova testa della lista di partenza).


\subsection{equal-list}

Verifica se due liste sono uguali.

\begin{lstlisting}
    fun equal_list(Empty,Empty) = true
    = |equal_list(_,Empty) = false
    = |equal_list(Empty,_) = false
    = |equal_list(nodo(x,l1),nodo(y,l2)) =
    = if x=y then equal_list(l1,l2) else false;
\end{lstlisting}    

\subsection{insert-sorted}

Permette di inserire un elemento in una lista effettuandolo in modo crescente.

\begin{lstlisting}
    fun insert_sorted(x,Empty) = nodo(x,Empty)
    = |insert_sorted(x,nodo(h,t)) =
    = if x<=h then nodo(x,nodo(h,t)) 
      else nodo(h,insert_sorted(x,t));
\end{lstlisting}

\subsection{is-sorted}

Verifica che la lista abbia gli elementi inseriti in modo crescente.

\begin{lstlisting}
    fun is_sorted(nodo(_,Empty)) = true
    = |is_sorted(nodo(x,nodo(h,t))) = 
    if x<=h then is_sorted(nodo(h,t)) else false;
\end{lstlisting}

\subsection{count-between}

Dati due estremi (a,b) si contano gli elementi compresi tra di loro.

\begin{lstlisting}
    fun count_between(_,_,Empty) = Empty
    = |count_between(a,b,nodo(x,l1)) = 
    if x>=a andalso x<=b then 
    nodo(x,count_between(a,b,l1)) 
    else count_between(a,b,l1);
\end{lstlisting}

\section{Alberi}

Gli alberi sono stati definiti nel modo seguente:
\begin{lstlisting}
    datatype 'a tree = Empty|node of 'a tree * int * 'a tree;
\end{lstlisting}

\subsection{count-nodes}

Conta il numero di nodi presenti nell'albero.

\begin{lstlisting}
    fun count_nodes Empty = 0
    = |count_nodes(node(lft,_,rht)) =
    = 1 + count_nodes lft + count_nodes rht;
\end{lstlisting}

\subsection{height}

Misura l'altezza dell'albero.

\begin{lstlisting}
    fun height Empty = 0
    = |height(node(lft,_,rht)) =
    1+max(height(lft),height(rht));
\end{lstlisting}

\subsection{find-element}

Verifica se un elemento è presente nell'albero.

\begin{lstlisting}
    fun find_element(_,Empty) = false
    = |find_element(x, node(lft,y,rht)) =
    = if x=y then true 
    else find_element(x,lft) orelse find_element(x,rht);

\end{lstlisting}

\subsection{sum-tree}

Fa la somma di tutti i valori presenti nell'albero.

\begin{lstlisting}
    fun sum_tree(Empty) = 0
    = |sum_tree(node(lft,x,rht)) =
    x + sum_tree(lft) + sum_tree(rht);
\end{lstlisting}

\subsection{mirror}

Fa lo speculare dell'albero dato in input.

\begin{lstlisting}
    fun mirror Empty = Empty
    = |mirror(node(left,x,right)) =
    node(mirror(right),x,mirror(left));
\end{lstlisting}

\subsection{preorder}

Si effettui la visita in preorder di un albero.

\begin{lstlisting}
    fun preorder(Empty) = []
    = |preorder(node(left,x,right)) =
    [x] @ preorder(left) @ preorder(right);
\end{lstlisting}

\subsection{insert-bst}

Si effettui l'inserimento in un albero di ricerca binario.

\begin{lstlisting}
    fun insert_bst(x, Empty) = node(Empty,x,Empty)
    = |insert_bst(x,node(left,y,right)) =
    = if x<y then node(insert_bst(x,left),y,right)
    else node(left,y,insert_bst(x,right));
\end{lstlisting}

\subsection{count-only-child}

Si scriva una funzione che calcoli quante foglie hanno solo un nodo figlio.

\begin{lstlisting}
    fun count_only_child(node(Empty,_,Empty)) = 0
    = |count_only_child(node(left,x,Empty)) = 
    x + count_only_child(left)
    = |count_only_child(node(Empty,x,right)) = 
    x + count_only_child(right)
    = |count_only_child(node(left,x,right)) = 
    count_only_child(left) + count_only_child(right);
\end{lstlisting}

\subsection{build-max}

Si scriva una funzione che calcoli il valore massimo tra i sottoalberi e costruisca un albero di soli valori massimi.

\begin{lstlisting}
    fun build_max(Empty) = Empty
    |build_max(node(left,x,right))=
    let
    val max_left = build_max(left)
    val max_right = build_max(right)
    val max_val = max(max_left,max_right)
    in
    node(max_left,max_val,max_right)
    end;
\end{lstlisting}

\subsection{maxLeaf}

Dato il tipo di albero (per i prossimi esercizi sarà lo stesso):
\begin{lstlisting}
    datatype albero = vuoto|nodo of int*albero*albero;
\end{lstlisting}
Si scriva una funzione che dato un albero calcoli il valore massimo presente.

\begin{lstlisting}
    fun maxLeaf(vuoto) = 0
    = |maxLeaf(nodo(x,vuoto,vuoto)) = x
    = |maxLeaf(nodo(x,left,right))=
    = let
    = val max_left = maxLeaf(left)
    = val max_right = maxLeaf(right)
    = in
    = if max_left>max_right then max_left 
    else max_right
    = end;
\end{lstlisting}

\subsection{treeMap}

Si scriva una funzione che sia map (funzione di ordine superiore per le liste), ma per gli alberi binari.

\begin{lstlisting}
    fun treeMap(f, vuoto) = vuoto
    = |treeMap(f, nodo(x,left,right))=
    = let
    = val treeMapLeft = treeMap(f,left)
    = val treeMapRight = treeMap(f, right)
    = in
    = nodo(f(x),treeMapLeft,treeMapRight)
    = end;
\end{lstlisting}

\subsection{sameStructure}

Si scriva una funzione che dati due alberi verifichi che entrambi hanno la stessa struttura.

\begin{lstlisting}
    fun sameStructure(vuoto, vuoto) = true
      | sameStructure(_, vuoto) = false
      | sameStructure(vuoto, _) = false
      | sameStructure(nodo(_, left1, right1), nodo(_, left2, right2)) =
          let
            val structLeft = sameStructure(left1, left2)
            val structRight = sameStructure(right1, right2)
          in
            structLeft andalso structRight
          end;

\end{lstlisting}

\subsection{biggest}

Si scriva una funzione che dati due alberi verifichi quale dei due ha la somma di valori maggiore.

\begin{lstlisting}
    fun biggest(vuoto,vuoto) = vuoto
    = |biggest(a1,vuoto) = a1
    = |biggest(vuoto,a2) = a2
    = |biggest(nodo(x,left1,right1),nodo(y,left2,right2)) =
    = let
    = fun sumTree(vuoto) = 0
    = |sumTree(nodo(x,left,right)) = x + sumTree(left) + sumTree(right)
    = val sumA1 = x + sumTree(left1) + sumTree(right1)
    = val sumA2 = y + sumTree(left2) + sumTree(right2)
    = in
    = if sumA1>sumA2 then nodo(x,left1,right1) else nodo(y,left2,right2)
    = end;
\end{lstlisting}

\subsection{plus}

Si scriva una funzione che aggiunga un certo intero n a un albero.

\begin{lstlisting}
    fun plus(vuoto,_) = vuoto
    = |plus(nodo(x,left,right),n) =
    = let
    = val plusLeft = plus(left,n)
    = val plusRight = plus(right,n)
    = in
    = nodo(x+n,plusLeft,plusRight)
    = end;
\end{lstlisting}

\subsection{remove-leaves}

Si scriva una funzione che rimuova tutte le foglie da un albero.

\begin{lstlisting}
    fun remove_leaves(vuoto) = vuoto
    = |remove_leaves(nodo(_,vuoto,vuoto)) = vuoto
    = |remove_leaves(nodo(x,left,right)) = 
    nodo(x,remove_leaves(left),remove_leaves(right));
\end{lstlisting}

\subsection{treeCount}

Si scriva una funzione che prenda in ingresso un albero e una funzione da interi a booleani e restituisca il numero di nodi per cui la condizione della funzione a interi è vera.

\begin{lstlisting}
fun treeCount(vuoto,_) = 0
= |treeCount(nodo(x,left,right),f) =
= let
= val treeLeft = treeCount(left, f)
= val treeRight = treeCount(right, f)
= val sum = treeLeft + treeRight
= in
= if f(x) = true then sum + 1 else sum
= end;
val treeCount = fn : albero * (int -> bool) -> int
\end{lstlisting}