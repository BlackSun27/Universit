\chapter{Prolog}

\subsection{sublist}

Si scriva un predicato che date due liste (L1 e L2) verifichi se L2 è sottolista di L1.

Esempio:

\begin{lstlisting}
    ?- sublist([a,b,c],[a,b,z,x,y,a,b,c,d]).
    true
\end{lstlisting}

\begin{lstlisting}
    sublist([],_). %caso base: lista vuota contenuta in L2
    sublist([H|L1],[H|L2]) :- 
    %caso 1: testa uguale tra le liste
    sublist(L1,L2).
    sublist([H1|L1],[H2|L2]) :- 
    %caso 2: le teste non coincidono, 
    %ma si fa proseguire tutta la lista finche' 
    %non viene fatta la prima chiamata 
    %alla sublist di sopra.
        H1 \= H2,
        atom(H2),
        sublist([H1|L1],L2).
\end{lstlisting}

\subsection{is-sorted}

Verifica se una lista è ordinata in ordine crescente.

\begin{lstlisting}
    is_sorted([]). %caso base 1: lista vuota 
    is_sorted([_]). 
    %caso base 2: lista con un solo elemento e' gia' ordinata
    is_sorted([X, Y|Xs]):-
        X=<Y,
        is_sorted([Y|Xs]). 
        %caso base 3: lista ordinata in ordine crescente
\end{lstlisting}

\subsection{invert}

Inverte gli elementi presenti in una lista.

\begin{lstlisting}
    invert([],[]). %caso 1: lista vuota
    invert([X|Xs], Rev):- %caso 2: inversione lista
        invert(Xs, T),
        append(T, [X], Rev).
\end{lstlisting}

\subsection{extremes}

Si scriva un predicato che trovi gli estremi di una lista.

\begin{lstlisting}
    extremes([],_,_). %caso 1: lista vuota
    extremes([X], X, X). %caso 2: un solo elemento
    extremes([F|Xs], F, L):- 
    %caso 3: verifica degli estremi
        last(Xs,L). 
        %predicato gia' esistente in Prolog last/2
\end{lstlisting}

\subsection{max}

Predicato che calcoli il massimo numerico di una lista.

\begin{lstlisting}
    max([],_). %caso 1: lista vuota
    max([X],X):- %caso 2: un solo elemento
        number(X).
    max([X|Xs], Y):- 
    %caso 3: verifica il maggiore se il primo e' numerico
        number(X), 
        max(Xs,Y),
        X>=Y.
    max([_|Xs], Y):- 
    %caso 4: verifica qual e' il maggiore se il primo non e' numerico
        max(Xs,Y).
\end{lstlisting}

\subsection{is-monotonic}

Si verifichi se una lista è monotona (crescente o decrescente).

\begin{lstlisting}
    is_monotonic([H|T]):-
        is_decrescent([H|T]).
    
    is_monotonic([H|T]):-
        is_crescent([H|T]).
    
    is_decrescent([]).
    is_decrescent([_]).
    is_decrescent([H,R|T]):-
        H>=R,
        is_decrescent([R|T]).
    
    is_crescent([]).
    is_crescent([_]).
    is_crescent([H,R|T]):-
        H=<R,
        is_crescent([R|T]).
\end{lstlisting}

\subsection{max}

Si calcoli la somma di tutti i numeri presenti nella lista.

\begin{lstlisting}
    max([],0). %caso base: lista vuota
    max([X],X):- 
        number(X). 
    	%caso 1: solo un elemento, dev'essere numerico
    
    max([H|T],S):- %caso ricorsivo, se numerico
        number(H),
        max(T,S1),
        S is H+S1.
    
    max([_|T],S):- %caso ricorsivo, se !number(H)
        max(T,S).
\end{lstlisting}

\subsection{almost-equal}

Si scriva un predicato che prese due liste verifichi se hanno al più una differenza, esempio:

\begin{lstlisting}
    ?- almost_equal([a,b,c],[a,7,c]).
    true

    ?- almost_equal([a,b,c],[a,b,c]).
    true

    ?- almost_equal([a,b,c],[a,7,8]).
    false
\end{lstlisting}

\begin{lstlisting}
    almost_equal(L1, L2) :-
        length(L1, A1),
        length(L2, A2),
        % Controlla che le liste abbiano la stessa lunghezza
        A1 =:= A2,
        % Conta le differenze tra le liste
        count_difference(L1, L2, 0). 
        
    count_difference([], [], _). % Caso base: fine delle liste,
                                 %accetta 0 o 1 differenze
    count_difference([H1|T1],[H1|T2],C):-
        count_difference(T1,T2,C). 
        %teste uguali, si continua la ricorsione
    
    count_difference([H1|T1],[H2|T2],C):-
        %teste diverse: si verifica che ci sia solo una differenza
        H1\=H2,
        C < 1,
        NC is C+1,
        count_difference(T1,T2,NC).
\end{lstlisting}

\subsection{value-length}

Si scriva una funzione che calcoli la lunghezza di una lista.

\begin{lstlisting}
    value_length([], 0). %caso base: 0 elementi
    value_length([_|END], A):- %caso 1: conta tutti gli
    %elementi fino alla fine
        value_length(END, A1),
        A is A1 + 1.
\end{lstlisting}

\subsection{last-element}

Si scriva un predicato che individui l'ultimo elemento di una lista.

\begin{lstlisting}
    last_element([X], X). %caso base: l'ultimo elemento corrisponde 
    %all'ultimo elemento

    last_element([_|T],X):- %caso 1: scorro la lista fino all'ultimo
    					%elemento
    last_element(T,X).
\end{lstlisting}

\subsection{is_member}

Predicato che verifichi se un elemento è presente nella lista.

\begin{lstlisting}
    is_member([X],X). %caso 1: X e' l'elemento di [X]
    is_member([X|_],X). %caso 2: X e' l'elemento in testa alla lista
    is_member([_|T], X):- 
    %caso 3: scorro la lista fin quando non trovo T=X
    is_member(T,X).
\end{lstlisting}

\subsection{concatenate}

Si scriva un predicato per concatenare 2 liste.

\begin{lstlisting}
    concatenate([],List,List). %caso base: lista vuota concat lista = lista
    concatenate(L1, L2, Result):- %caso 1: si concatena
    append(L1,L2,Result).
\end{lstlisting}    

\subsection{reverse-list}

Si scriva un predicato che inverta una lista.

\begin{lstlisting}
    reverse_list([],[]). %caso -1: lista vuota gia' reversed
    reverse_list([X], [X]). 
    %caso base: 1 elemento = lista gia' reverse
    reverse_list([X|Xs], R):- %caso ricorsivo: s'inverte la lista 
        reverse_list(Xs,T),
        append(T,[X],R).
\end{lstlisting}



\subsection{sum-list}

Si scriva un predicato che faccia la somma di tutti gli elementi in una lista.

\begin{lstlisting}
    sum_list([],0). %caso base: 0 elementi valore = 0
    sum_list([X],X). %caso 1: 1 elemento = l'elemento
    sum_list([H|T], S):-
        sum_list(T,S1),
        S is S1 + H.
\end{lstlisting}

\subsection{max-list}

Si scriva un predicato che trovi il massimo in una lista.

\begin{lstlisting}
max_list([H|T], Max) :-
    max_list(T, H, Max). %max_list/3 mi serve per iniziare la ricorsione

max_list([], Max, Max). %caso base: il massimo della lista e' stato trovato

max_list([H|T],Max0,Max):-
    H>Max0, %la testa e' maggiore del massimo visitato finora, ricorsione
    max_list(T,H,Max).

max_list([H|T],Max0,Max):-
    H<Max0, 
    max_list(T,Max0,Max). %si continua la ricorsione finche' non termina
\end{lstlisting}

\subsection{remove-element}

Si scriva un predicato che rimuova un elemento da una lista.

\begin{lstlisting}
remove_element(_, [], []). %caso base: lista vuota, nulla da rimuovere
remove_element(E, [H|T], [H|L]):- 
    E \= H, 
    %caso in cui E e' diverso da H, gia' e' presente nella lista L
    remove_element(E, T, L).

remove_element(E, [H|T], L):-
    E == H, 
    %caso in cui E e' uguale a H, non si aggiunge alla lista L
    remove_element(E, T, L).
\end{lstlisting}

\subsection{split-list}

Predicato che data una certa lunghezza divide la lista in 2 liste diverse (la lunghezza della lista di partenza parte sempre da 0).

\begin{lstlisting}
split_list(L, N, L1, L2) :- % Predicato principale
    length(L, Len),
    Len >= N, % Assicura che la lista abbia almeno N elementi
    split_list_helper(L, N, 0, L1, L2). % Chiama l'helper con il contatore iniziale

split_list_helper([], _, _, [], []). % Caso base: lista vuota

split_list_helper([H|T], N, P, [H|L], R) :- % Caso 1: aggiungi a sinistra
    P < N, % Controlla se il contatore e' minore di N
    P1 is P + 1, % Incrementa il contatore
    split_list_helper(T, N, P1, L, R).

split_list_helper([H|T], N, P, L, [H|R]) :- % Caso 2: aggiungi a destra
    P >= N, % Controlla se il contatore ha raggiunto N
    split_list_helper(T, N, P, L, R).
\end{lstlisting}

\subsection{split-even-odd}

Si scriva un predicato che divida una lista in due liste, rispettivamente, una di numeri pari e l'altra di numeri dispari.

\begin{lstlisting}
split_even_odd([],[],[]). %liste vuote, se la lista e' vuota
split_even_odd([H|T], [H|E], O):- %divisione in lista pari
    0 is mod(H,2),
    split_even_odd(T, E, O).

split_even_odd([H|T],E,[H|O]):- %divisione in lista dispari
    1 is mod(H,2),
    split_even_odd(T, E, O).
\end{lstlisting}

\subsection{is-palindrome}

Predicato che verifichi se una lista è palindroma.

\begin{lstlisting}
is_palindrome([]).
is_palindrome([_]).
is_palindrome([H|T]):-
    append(H1, [H], T),
    is_palindrome(H1).
\end{lstlisting}

Questo predicato funziona per com'è definito il predicato append, come ogni predicato può essere vero o falso e se H1, [H] e T sono uguali tra loro, allora la lista di partenza è palindroma.

\subsection{perm}

Si scriva un predicato che calcoli tutte le permutazioni di una lista.

\begin{lstlisting}
%Ragionamento per le permutazioni in una lista

%caso base: lista vuota -> lista vuota
perm([],[]).

%caso 1: riordino la lista per ricoprire tutte le forme che puo' avere la lista
%-> probabilmente serve un helper/3
perm([X|Y],Z):-
    perm(Y,W),
    helper(X,Z,W).

%caso base helper/3: se la lista di partenza ha la stessa testa della lista
%allora costruisco solo la lista della coda
helper(X,[X|T],T). 

%caso ricorsivo: costruisco la lista risultante
helper(X,[F|R],[F|S]):-
    helper(X,R,S).
\end{lstlisting}